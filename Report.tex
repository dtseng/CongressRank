\documentclass[11pt]{article}
\usepackage{amsfonts}
\usepackage{graphicx}
\usepackage{amsmath,textcomp,amssymb,geometry,graphicx,enumerate}
\usepackage{amssymb}
\usepackage{algorithm} % Boxes/formatting around algorithms
\usepackage[noend]{algpseudocode} % Algorithms
\usepackage[usenames,dvipsnames]{color}
\usepackage{listings}
\definecolor{MyDarkGreen}{rgb}{0.0,0.4,0.0}
\lstloadlanguages{python}
\lstset{language=python,
        frame=single,
        basicstyle=\small\ttfamily,
        keywordstyle=[1]\color{Blue}\bf,
        keywordstyle=[2]\color{Purple},
        keywordstyle=[3]\color{Red},
        identifierstyle=,
        commentstyle=\usefont{T1}{pcr}{m}{sl}\color{MyDarkGreen}\small,
        stringstyle=\color{Purple},
        showstringspaces=false,
        tabsize=2,
        morecomment=[l][\color{Blue}]{...},
        numbers=left,
        numberstyle=\tiny,
        breaklines=true,
%        title=\lstname                             %commented out since we're not providing .java files this time -NF
}

\begin{document}

\title{CongressRank: A PageRank Application}
\author{Chandler Chen, David Tseng}
\date{\today}
\maketitle

\section*{Introduction}
When a bill is introduced to Congress, the original proponent is usually called the \textit{sponsor}. Congress members who support the bill can sign onto the bill as a \textit{cosponsor}. The political community places great emphasis on cosponsorships, as it is believed that a bill with more cosponsorships demonstrates a greater support base. Thus, the number of cosponsorships of a bill is an indicator of the popularity of the bill among Congress members. Since a Congress member will only agree to cosponsor a sponsor's bill if he or she agrees with the bill's contents, a cosponsorship can also symbolize the cosponsor's support for and agreement with the sponsor. We believe that by analyzing the network of cosponsorships, we can have a better understanding of the power dynamics and unity within Congress. 

In this project, we used PageRank to analyze the cosponsorship graph. PageRank is a ranking algorithm developed to find the stationary distribution of a Markov Chain, where a crawler visits other nodes (where neighboring nodes are chosen uniformly at random) for many iterations. As the number of iterations increases, the percentage of visits a node receives converges to the graph's stationary distribution, and nodes are ranked based on the number of visits they receive. 

% include graphic of edge pointing from cosponsor to sponsor. 


\section*{Methods}
In our project, we represented each Congress member as a node. If member $u$ cosponsors member $v$, then a directed edge is placed from $u$ to $v$ on the graph. If $u$ cosponsors $v$ on $x$ different bills, there will be $x$ distinct directed edges from $u$ to $v$. All Representatives, Senators, and non-voting members are represented the same way. 

We decided to use data from the 114th Congress (2015-2017), since this was the most recent Congress that completed its term. Although the 114th Congress had 100 Senators, 435 Representatives, and 6 non-voting members, there were 7 additional members who left Congress before their term ended. Therefore, we included a total of 548 members (nodes) in our graph. To extract the bill and member data, we used ProPublica's Congress API, which presented us the data in a convenient JSON format. 



%http://tex.stackexchange.com/questions/160540/how-to-number-figures-continuously-in-documentclassarticle



\textbf{Pseudocode}\\
\begin{algorithmic}[1]
\Procedure{createGraph}{bills}
\State graph = empty dictionary, mapping each of the 548 members to an empty list.   This will be a graph mapping a member to a list of people he/she cosponsored. 
\For{bill in bills}
	\State $u$ = bill[sponsor]
	\For {cosponsor in bill[cosponsors]}
		\State graph[cosponsor].append()
	\EndFor 


\EndFor







\EndProcedure
\end{algorithmic}















\end{document}	