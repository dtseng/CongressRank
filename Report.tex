\documentclass[11pt]{article}
\usepackage{amsfonts}
\usepackage{graphicx}
\usepackage{amsmath,textcomp,amssymb,geometry,graphicx,enumerate}
\usepackage{amssymb}
\usepackage[usenames,dvipsnames]{color}
\usepackage{listings}
\definecolor{MyDarkGreen}{rgb}{0.0,0.4,0.0}
\lstloadlanguages{python}
\lstset{language=python,
        frame=single,
        basicstyle=\small\ttfamily,
        keywordstyle=[1]\color{Blue}\bf,
        keywordstyle=[2]\color{Purple},
        keywordstyle=[3]\color{Red},
        identifierstyle=,
        commentstyle=\usefont{T1}{pcr}{m}{sl}\color{MyDarkGreen}\small,
        stringstyle=\color{Purple},
        showstringspaces=false,
        tabsize=2,
        morecomment=[l][\color{Blue}]{...},
        numbers=left,
        numberstyle=\tiny,
        breaklines=true,
%        title=\lstname                             %commented out since we're not providing .java files this time -NF
}

\begin{document}

\title{CongressRank: A PageRank Application}
\author{Chandler Chen, David Tseng}
\date{\today}
\maketitle

\section*{Introduction}
When a bill is introduced to Congress, the original proponent is usually called the \textit{sponsor}. Congress members who support the bill can sign onto the bill as a \textit{cosponsor}. The political community tends to place emphasis on cosponsorships, as it is believed that a bill with more cosponsorships demonstrates a greater support base. Thus, the number of cosponsorships of a bill is an indicator of the popularity of the bill among Congress members. Since a Congress member will only agree to cosponsor a sponsor's bill if he or she agrees with the bill's contents, a cosponsorship can also symbolize the cosponsor's support for and agreement with the sponsor. We believe that by analyzing the network of cosponsorships, we can have a better understanding of the power dynamics and unity within Congress. 


% include graphic of edge pointing from cosponsor to sponsor. 


\section*{Methods}
We decided to use bill data from the 114th Congress (2015-2017), since this was the most recent Congress with two years of data. In the cosponsorship graph, each node represented a Congress member, and Congress member $u$'s cosponsorship of member $v$ is represented as a directed edge from $u$ to $v$. 





% talk about number of Congress members. There are more members because we are including the nonvoting members. 

%http://tex.stackexchange.com/questions/160540/how-to-number-figures-continuously-in-documentclassarticle



















\end{document}	